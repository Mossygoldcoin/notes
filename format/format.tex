% Korean
\usepackage{kotex}
%%%%%%%%%%%%%%%%%%%%%%%%%%%%%%%
\usepackage{titlesec}
\usepackage{color}
\usepackage{graphicx}
\usepackage[english]{babel}
\usepackage{xstring}
\usepackage{fancyvrb}
\usepackage{setspace}
%%%%%%%%%%%%%%%%%%%%%%%%%%%%%%%
% Appendix
\usepackage[toc,page]{appendix}
% Theorem
\usepackage[framemethod=TikZ]{mdframed}
\usepackage{amssymb,amsmath,amsthm}
\usepackage{thmtools}
\usepackage[most]{tcolorbox}
% Font
\usepackage{cabin}
\usepackage{anyfontsize}
% Book
\usepackage{cantarell}
% Quotation
\usepackage{csquotes}
% Hyperlink
\usepackage{hyperref}
\hypersetup{
  filecolor=magenta,      
  urlcolor=cyan,
}
% Code
\usepackage{listings}
% Diagram
\usepackage{tikz}
\usepackage{tikz-cd}
% Math
\usepackage{amsmath}
\usepackage{amssymb}
\usepackage{amsfonts}
\usepackage{mathtools}
% List
\usepackage{enumitem}
% Table
\usepackage{booktabs}
\usepackage{multirow}
\usepackage{tabularx}
\usepackage{makecell}
\usepackage{longtable}
\usepackage{array}
\usepackage{threeparttable}
\usepackage{threeparttablex}
\usepackage{caption}
% Subfile
\usepackage{subfiles}
% Epigraph
\usepackage{epigraph}
% Index
\usepackage{imakeidx}
\usepackage{etoolbox}
\makeindex[intoc, title=Index]
% Quote
\usepackage{upquote}
% Ruby
\usepackage{ruby}

\usepackage{hyperref}
\usepackage{cleveref}

\renewcommand{\normalsize}{\fontsize{11pt}{13.2pt}\selectfont}

% ------ Default Title -----
\definecolor{wikipediadarkblue}{rgb}{0.023, 0.270, 0.676}
\providecommand{\pdftitle}{notes}
\hypersetup{
  colorlinks,
  citecolor=black,
  filecolor=black,
  linkcolor=wikipediadarkblue,
  urlcolor=wikipediadarkblue,
  pdftitle={\pdftitle},
  pdfauthor={Swion Yun}
}
% --------------------------

\makeatletter
\let\oldauthor\author
\renewcommand{\author}[1]{
    \oldauthor{#1}
    \gdef\printauthor{#1}
}
\makeatother

% -------- Variable --------
\newif\iftoc
\tocfalse
\newif\ifinappendix
\inappendixfalse

\makeatletter
\def\strip@parenthesis#1{\expandafter\strip@paren@aux#1)\@nil}
\def\strip@paren@aux(#1)\@nil{#1}

\def\thmt@setheadstyle#1{%
  \thmt@style@headstyle{%
    \def\NAME{\the\thm@headfont ##1}%
    \def\NUMBER{\bgroup\@upn{##2}\egroup}%
    \def\NOTE{\if=##3=\else\bgroup\thmt@space\the\thm@notefont(##3)\egroup\fi}%
  }%
  \def\thmt@tmp{}
  \@onelevel@sanitize\thmt@tmp
  \ifcsname thmt@headstyle@\thmt@tmp\endcsname
    \thmt@style@headstyle\@xa{%
      \the\thmt@style@headstyle
      \csname thmt@headstyle@#1\endcsname
    }%
  \else
    \thmt@style@headstyle\@xa{%
      \the\thmt@style@headstyle
      #1
    }%
  \fi
}
% --------------------------

% -------- Spacing ---------
\usepackage{setspace}
\setstretch{1.1}
% --------------------------

% ---------- Title ---------
\makeatletter
\def\@maketitle{
  \newpage
  \null
  \vskip 2em
  \begin{center}
    \let\footnote\thanks
    {\huge\bfseries \@title \par}
    \vskip 1.5em
    {\large \@author\par}
    \vskip 1em
    {\large \@date\par}
  \end{center}
  \par
\vskip 1.5em}
\makeatother

\titleformat{\section}
{\normalfont\Large\bfseries}
{\thesection}
{1em}
{}
% --------------------------

% -------- Geometry --------
\usepackage[margin=1in, marginparwidth=2cm, headheight=15pt]{geometry}
% --------------------------

% -------- Titlesec --------
\usepackage{titlesec}
\newif\iftoc

% Formatting of part
\titleformat
{\part}
[display]
{\cabin\bfseries\LARGE\scshape}
{\centering\LARGE Part \thepart}
{10mm}
{\centering\Huge}
[
\thispagestyle{empty}
]
\titlespacing*{\part}{0mm}{30mm}{30mm}
\titleclass{\part}{top}
% --------------------------

% --------- Header ---------
\usepackage{fancyhdr}
\pagestyle{fancy}
\fancyhf{}
\fancyfoot[C]{\thepage}
\fancyhead[LE,RO]{\scshape\nouppercase{\leftmark}}

\fancypagestyle{plain}{
  \fancyhf{}
  \fancyfoot[C]{\thepage}
  \renewcommand{\headrulewidth}{0pt}
}
% --------------------------

% --------- Index ----------
\newcounter{indexcounterlabel}
\newcommand{\indexlabel}[1]{\label{indexentry:#1}}
\let\oldindex\index

\renewcommand{\index}[1]
{
  \stepcounter{indexcounterlabel}
  \indexlabel{\theindexcounterlabel}
  \oldindex{#1|hyperref[indexentry:\theindexcounterlabel]}
}

\newcommand{\definitionindex}{\newpage\printindex}
% --------------------------

% ----------- Box ----------

\definecolor{color-red}{HTML}{ba4562}
\definecolor{color-blue}{HTML}{3740c8}
\definecolor{color-green}{HTML}{3b7d02}
\definecolor{color-yellow}{HTML}{f0a500}
\definecolor{color-purple}{HTML}{7a00a3}
\definecolor{color-orange}{HTML}{ff7f00}
\definecolor{color-cyan}{HTML}{00a3a3}
\definecolor{color-pink}{HTML}{ff007f}

\tcbset{
  style-box/.style={
    enhanced jigsaw,
    breakable,
    sharp corners,
    toprule=0pt, rightrule=0pt, bottomrule=0pt, leftrule=2.5pt,
    colback=#1!12!black!7, colframe=#1!80, coltitle=#1!80,
  },
}

\tcbset{
  style-mini-box/.style={
%     breakable,
    sharp corners,
    toprule=0pt, rightrule=0pt, bottomrule=0pt, leftrule=2.5pt,
    colback=#1!12!black!7, colframe=#1!80, coltitle=#1!80,
  },
}

\newcommand{\tbox}[2]{\begin{tcolorbox}
    [style-box=color-blue]
    \textbf{#1}
    \vspace{0.2em}\\
    {#2}
\end{tcolorbox}}

\newcommand{\ttbox}[2]{\begin{tcolorbox}
    [style-box=color-blue,left skip=10mm,right skip=10mm]
    \textbf{#1}
    \vspace{0.5em}\\
    {#2}
\end{tcolorbox}}

\newcommand{\tminibox}[2]{\begin{tcolorbox}
    [style-mini-box=color-blue]
    \textbf{#1}
    \vspace{0.5em}\\
    {#2}
\end{tcolorbox}}

\definecolor{darkgreen}{HTML}{20660a}
\definecolor{lightgreen}{HTML}{ebffeb}
\definecolor{darkred}{HTML}{8B0000}
\definecolor{lightred}{HTML}{FFEBEB}
\definecolor{darkorange}{HTML}{D35400}
\definecolor{lightorange}{HTML}{FFE5D5}
\definecolor{darkblue}{HTML}{002366}
\definecolor{lightblue}{HTML}{E3F2FD}
\definecolor{teal}{HTML}{008080}
\definecolor{lightteal}{HTML}{DFF5F2}
\definecolor{darkpurple}{HTML}{4B0082}
\definecolor{lightpurple}{HTML}{F3E5F5}
\definecolor{darkpink}{HTML}{C71585}
\definecolor{lightpink}{HTML}{FFD9EC}
\definecolor{darkyellow}{HTML}{B8860B}
\definecolor{lightyellow}{HTML}{FFF9C4}
\definecolor{darkcyan}{HTML}{006666}
\definecolor{lightcyan}{HTML}{E0FFFF}
\definecolor{darkbrown}{HTML}{5D4037}
\definecolor{lightbrown}{HTML}{D7CCC8}
\definecolor{darkviolet}{HTML}{6A1B9A}
\definecolor{lightviolet}{HTML}{F3E5F5}

\mdfdefinestyle{greenbox}{
  backgroundcolor=lightgreen,
  linecolor=darkgreen,
  linewidth=2pt,
  topline=false,
  bottomline=false,
  rightline=false,
  innertopmargin=5pt,
  innerbottommargin=8pt,
  innerleftmargin=8pt,
  skipabove=10pt,
  skipbelow=10pt
}

\usepackage{ifthen}
\newcommand{\createtheoremstyle}[3]{
        \mdfdefinestyle{#1box}{
        backgroundcolor=#3,
        linecolor=#2,
        linewidth=2pt,
        topline=false,
        bottomline=false,
        rightline=false,
        innertopmargin=5pt,
        innerbottommargin=8pt,
        innerleftmargin=8pt,
        skipabove=10pt,
        skipbelow=10pt
    }

   \ifthenelse{\equal{#1}{definition}}{
        \declaretheoremstyle[
            headfont=\bfseries\color{#2},
            bodyfont=\normalfont,
            mdframed={style=#1box},
            headformat={\NAME~\NUMBER\NOTE\ifx##3\empty\else\llap{\index{##3}}\fi\text{.}},
            headpunct={},
            numbered=yes
        ]{#1style}
    }{
        \declaretheoremstyle[
            headfont=\bfseries\color{#2},
            bodyfont=\normalfont,
            mdframed={style=#1box}
        ]{#1style}
    }

    \declaretheorem[
        style=#1style,
        name={\MakeUppercase{#1}},
        numberwithin=chapter
    ]{#1}
}

\createtheoremstyle{definition}{darkblue}{lightblue}
\createtheoremstyle{example}{darkgreen}{lightgreen}
\createtheoremstyle{exercise}{darkred}{lightred}
\createtheoremstyle{remark}{darkpurple}{lightpurple}
\createtheoremstyle{note}{darkorange}{lightorange}
\createtheoremstyle{lemma}{darkpink}{lightpink}
\createtheoremstyle{corollary}{darkyellow}{lightyellow}
\let\proof\relax
\let\endproof\relax
\createtheoremstyle{proof}{teal}{lightteal}
\createtheoremstyle{theorem}{darkcyan}{lightcyan}
\createtheoremstyle{proposition}{darkbrown}{lightbrown}
\createtheoremstyle{conjecture}{darkviolet}{lightviolet}
% --------------------------

% -------- Display ---------
\titleformat
  {\chapter}
  [display]
  {\cabin}
  {}
  {2in}
  {
    \raggedleft
    \iftoc
      \vspace{2in}
    \else
      \ifinappendix
        {\LARGE\textsc{Appendix}~{\thechapter}}\\
      \else
        {\LARGE\textsc{Theme}~{\cantarell\thechapter}}\\ 
      \fi
      \fi
      \Huge\scshape\bfseries
    }
    [
    \vspace{-18pt}
    \rule{\textwidth}{0.1pt}
    \vspace{0.0in}
    ]
    \titlespacing{\chapter}
    {0pt}
    {
      \iftoc
        -103pt+1in
      \else
        -127pt+1in
      \fi
    }
    {0pt}
% --------------------------

% -------- Epigraph --------
\renewcommand\textflush{flushright}

\makeatletter
\newlength\epitextskip
\pretocmd{\@epitext}{\em}{}{}
\apptocmd{\@epitext}{\em}{}{}
\patchcmd{\epigraph}
	{\@epitext{#1}\\}
	{\vspace{-0.3in+20pt}\@epitext{#1}\\[\epitextskip]}
	{}
	{}
\makeatother

\setlength\epigraphrule{0pt}
\setlength\epitextskip{2ex}
\setlength\epigraphwidth{.6\textwidth}
\setlength\afterepigraphskip{30pt}
% --------------------------
