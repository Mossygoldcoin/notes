\documentclass[../note.tex]{subfiles}

\begin{document}

수업 시간에서 나온 조건 양화사와 전체 양화사의 예시를 살펴보고 뉘양쓰의 차이를 인지하여 보자.
\begin{example}
  \begin{align}
    \exists 0 \in \RR\; s.t.\; a+0 = 0+a = 0\quad \forall a \in \RR \label{eq:identity-element} \\
    \forall a \in \RR \quad \exisits -a \in \RR\; s.t.\; a+(-a) = (-a)+a = 0 \label{eq:inverse-element}
  \end{align}
  equation \eqref{eq:identity-element}은 덧셈의 항등원을 나타내고, equation \eqref{eq:inverse-element}은 덧셈의 역원을 나타낸다. 두 식을 자연어로 풀이하자면 다음과 같다:
  \begin{itemize}
    \item
      \eqref{eq:identity-element} : 모든 실수 $a$에 대하여 어떤 실수 $0$이 존재하며 $a+0 = 0+a = 0$이다.
    \item
      \eqref{eq:inverse-element} : 모든 실수 $a$에 대하여 어떤 실수 $-a$가 존재하며 $a+(-a) = (-a)+a = 0$이다.
  \end{itemize}
  두 식에서 등장한 $\forall a \in \RR$는 같은 `모든'이라는 의미를 내포한다면, 왜 equation\eqref{eq:identity-element}에서는 식 마지막에, equation \eqref{eq:inverse-element}에서는 식 처음에 위치하는지 의문이 들 수 있다. 1번째 식에서는 $a$의 값이 다른 조건에 영향을 주지 아니하는데 반해 2번째 식은 $\exists -a \in \RR$이 $a$에 종속된다. 한글로 의미를 전하자면, 각 식의 $\forall a\in\RR$는 1번째 식에서는 `임의의'로, 2번째 식에서는 `각각의'로 이해하자.
\end{example}

\section{명제와 집합}
사실 나는 고등학교 시절 명제\textsuperscript{proposition}에 대하여 배울 때 코로나에 감염되어 수업을 듣지 못하였다. --- 다만 정보에서도 중요한 내용인 만큼 수학 시간 외에도 배울 기회는 있었다. --- 때문에 명제에 대하여 수학 수업시간에 배운 적이 없는데, 명제의 중요함을 느끼고 다시 한번 정리하고자 한다.

\begin{definition}[명제]
  명제는 참\textsuperscript{true}인지 거짓\textsuperscript{false}인지 판정 가능한 문장이다. 명제는 다음과 같은 성질을 가진다:
  \begin{itemize}
    \item
      참인지 거짓인지 판정 가능해야 한다.
    \item
      참인지 거짓인지 판정 가능해야 한다.
  \end{itemize}
\end{definition}
즉, 명제는 조건\textsuperscript{condition}에 대 결과\textsuperscript{conclusion}의 참과 거짓을 판별한다. 조건 $p$와 결과 $q$에 대하여 일반적으로 다음과 같이 표현한다:
\begin{equation}
  p \longrightarrow q
\end{equation}

조건에 대한 결과가 참인 경우(참임을 검증 가능한 경우)에는 다음과 같이 표시한다:
\begin{equation}
  p \Longrightarrow q
\end{equation}

\begin{definition}[동형]
  명제 $p \Longrightarrow q$가 성립한다면, 이를 동형\textsuperscript{equivalences, isomorphism}이라 한다. $p$와 $q$에 대한 동형은 다음과 같이 표현하다:
  \begin{equation}
    p \Longleftrightarrow q
  \end{equation}
  이는 $p$와 $q$가 동일한 진리값을 가지면 다음을 만족한다:
  \begin{align}
    p \Longrightarrow q \\
    q \Longrightarrow p
  \end{align}
\end{definition}

추가로, $q \longrightarrow p$를 $p \longrightarrow q$에 대한 역\textsuperscript{converse}이라 하며, 진리값이 역인 $\not p$와 $~p$를 $p$에 대한 not이라고 한다.

\begin{definition}
  $p \longrightarrow q$에 대하여 $~q \longrightarrow ~p$를 대우\textsuperscript{contrapositive}이라 한다. 대우는 진리값이 일치하여 $p \Longrightarrow q$인 경우 대우 또한 참이다.
\end{definition}

\section{유리함수}
추가로 유리함수라는 단어가 종종 등장하지만 뜻이 기억나지 아니하여 적어둔다. 유리함수는 간단히 분모에 x가 포함된 함수로 기억하자:
\begin{example}[유리함수 예시]
  \item $y = \frac{2x}{x-3}$
\end{example}

\chapter{극한}
\section{들어가며}
드디어 수학 표현이 끝났다, 야호! 해당 대학수학 1(몇몇 학생이 혼돈이 있었던 듯 한데, 미분적분학이라 부르면 될 것을 왜 대학에선 대학수학으로 이름 지었는지 의문이다, 특별한 뜻이 있겠지.) 강좌는 미분적분학을 배우는 것을 목표로 하지만, 컴퓨터공학부 학생들만을 대상으로 하기에 엄밀함이 완전치 아니하다. (교수님도 강조하는 부분이다.) 때문에 추가로 엄밀한 증명이 필요한 경우, 또는 내가 궁금해서 근질거릴 때는 해당 노트에 적고자 한다.

\section{극한}
극한은 어떤 값에 가까워지는 것을 의미한다. 수업시간에 다루는 극한의 정의는 다음과 같다:
\begin{definition}[극한]
  $a$가 적당한 구간 I에 대하여 $a \in I$를 제외한 부분에 대한 함수값이 정의되어 있다. 이 때 함수 $f(x)$가 $x$가 $a$에 가까워질 때 $L$에 가까워진다면, 다음과 같이 표현한다:
  \begin{equation}
    \lim_{x \to a} f(x) = L\quad (\text{or}\; f(x) \to L\; \text{as}\; x \to a)
  \end{equation}
  이 경우 `$x$가 $a$에 한없이 가까워질 때 함수 $f$의 극한\textsuperscript{limit}은 $L$이다'라고 표현한다.
\end{definition}
함수 $f(x)$에 대한 극한 값은 특정 $x$에 대한 함숫값을 구하는 것이 아니다. 때문에, $f(a)$의 값이 정의되지 아니하더라도, $\lim_{x \to a} f(x)$의 값은 정의될 수 있다.

추가로 좌극한과 우극한도 정의한다.

\begin{definition}[좌극한과 우극한]
  $a$가 적당한 구간 I에 대하여 $a \in I$를 제외한 부분에 대한 함수값이 정의되어 있다. 이 때 함수 $f(x)$가 $x$가 $a$에 왼쪽에서 가까워질 때 $L$에 가까워진다면, 다음과 같이 표현한다:
  \begin{equation}
    \lim_{x \to a^-} f(x) = L\quad (\text{or}\; f(x) \to L\; \text{as}\; x \to a^-)
  \end{equation}
  이 경우 `$x$가 $a$에 한없이 왼쪽에서 가까워질 때 함수 $f$의 좌극한은 $L$이다'라고 표현한다.

  마찬가지로, $a$가 적당한 구간 I에 대하여 $a \in I$를 제외한 부분에 대한 함수값이 정의되어 있다. 이 때 함수 $f(x)$가 $x$가 $a$에 오른쪽에서 가까워질 때 $L$에 가까워진다면, 다음과 같이 표현한다:
  \begin{equation}
    \lim_{x \to a^+} f(x) = L\quad (\text{or}\; f(x) \to L\; \text{as}\; x \to a^+)
  \end{equation}
  이 경우 `$x$가 $a$에 한없이 오른쪽에서 가까워질 때 함수 $f$의 우극한은 $L$이다'라고 표현한다.
\end{definition}
좌극한과 우극한 역시 $f(a)$의 값은 정의되지 아니하여도 무관하며, 좌극한과 우극한은 극한의 서브적인 조건이다.
\begin{theorem}
  \label{theorem:relation-limit-one-side-limit}
  좌극한과 우극한이 $L$로 동일한 것은 극한이 $L$인 것과 동형이다. i.e.,
  \begin{equation}
    \lim_{x \to a} f(x) \Longleftrightarrow \lim_{x \to a^-} f(x) = \lim_{x \to a^+} f(x) = L
  \end{equation}
\end{theorem}
극한과 좌극한 및 우극한을 구분하기 위해 극한을 two-side limit이라고 말하며, 좌극한과 우극한을 one-side limit이라고 말한다.

또한, 대우는 진리값이 동일함을 이용하면 좌극한과 우극한이 다를 경우 극한이 존재하지 않음을 알 수 있다.

가까워진다면? 그게 무엇인가? 거리가 0.001이면 가까운 것인가, 아니면 거리가 0.000001일 때야 비로소 가까운 것인가? 엄격하지 아니한 정의이기 때문에 모호하고 알송달송하다.
\begin{example}
  \label{example:limit-example}
  다음 예제를 살펴보자. figure \ref{fig:limit-example}은 $(\tan(x) - x)/{x^3}$의 그래프에 대한 samples=150의 그래프이다. --- samples는 domain에 대하여 값을 측정한 횟수를 의미한다. 다음에서는 구간 $[-8, 8]$을 150개의 구간으로 나눈 것이다. --- 이 그래프에서 $x$가 0에 가까워지면 함수값 $L$은 어떤 값이 될까?

  \begin{center}
  \begin{tikzpicture}
    \begin{axis}[
        axis lines = middle,
        xlabel = $x$,
        ylabel = $y$,
        xmin = -8, xmax = 8,
        ymin = -6, ymax = 10,
        grid = both,
        grid style = {line width=.1pt, draw=gray!10},
        major grid style = {line width=.2pt,draw=gray!50},
        minor tick num = 4,
        title = {$\frac{\tan(x) - x}{x^3}$}
    ]

    \addplot[domain=-10:10, samples=150, thick, blue] {(tan(deg(x)) - x)/x^3};

    % \addplot[mark=*, color=red] coordinates {(0,2) (1,6)};

    \end{axis}
    \label{fig:limit-example}
    \caption{예제 그래프}
  \end{tikzpicture}
  \end{center}

  눈으로 보기에는 0과 1 사잇값, 약 0.5정도가 $\lim_{x \to 0} (\tan(x) - x)/{x^3}$의 값이 될 듯하다. 계산기를 이용하여 값을 구해 보아도 $f(x) = (\tan(x) - x)/{x^3}$일 때, $f(0.1) = 0.33$, $f(0.01) = 0.33$, $f(0.001) = 0.33$이다. 비단, 축적이 더욱 큰 동일한 그래프를 살펴보자.

  \begin{center}
  \begin{tikzpicture}
    \begin{axis}[
        axis lines = middle,
        % xlabel = $x$,
        % ylabel = $y$,
        xmin = -0.00002, xmax = 0.00002,
        ymin = -0.00002, ymax = 0.00002,
        grid = both,
        grid style = {line width=.1pt, draw=gray!10},
        major grid style = {line width=.2pt,draw=gray!50},
        minor tick num = 4,
        title = {$\frac{\tan(x) - x}{x^3}$}
    ]

    \addplot[domain=-0.00002:0.00002, samples=500, thick, blue] {(tan(deg(x)) - x)/x^3};

    % \addplot[mark=*, color=red] coordinates {(0,2) (1,6)};

    \end{axis}
    \label{fig:limit-example2}
    \caption{예제 그래프: 확대}
  \end{tikzpicture}
  \end{center}
  비록 \LaTeX의 tikz 모듈에서 해당 그래프가 잘 표현되지는 아니하였지만(x축 라벨링도 잘 못 되었다. domain은 -0.00002:0.00002이다.) 굉장히 작은 값에 대하여 크게 진동하고 있음을 알 수 있다. 사진을 캡처하고 파일로 옮기는 것은 생각보다 어려운(정확히는 귀찮은) 일이기에 geogebra link\footnote{https://www.geogebra.org/classic/u3cpa3qt}를 남긴다.
  
  해당 수식은 교재의 문제에서 가져왔는데, 마치 카오스를 처음 들었을 때와 같은 기분이 들었다. 이 예제가 극한에 대한 보다 엄격한 정의의 필요성을 보인다고 생각한다.
\end{example}

마음 같아서는 수업시간에 다루지 아니한 $\epsilon-\delta$ 논법에 대하여 정리하고프지만, 우선 ToDoo로 남겨 놓는다.

\section{무한대 발산}
\begin{definition}[무한대 발산]
  함수 $f(x)$가 $x$가 $a$에 가까워질 때 $f(x)$가 무한히 커진다면, 다음과 같이 표현한다:
  \begin{equation}
    \lim_{x \to a} f(x) = \infty
  \end{equation}
  이 경우 `$x$가 $a$에 한없이 가까워질 때 함수 $f$는 무한대로 발산한다'라고 표현한다.

  반대로 함수 $f(x)$가 $x$가 $a$에 가까워질 때 $f(x)$가 무한히 작아진다면, 다음과 같이 표현한다:
  \begin{equation}
    \lim_{x \to a} f(x) = -\infty
  \end{equation}
\end{definition}

\begin{example}
  figure \ref{fig:limit-example3}은 $1/x^2$의 그래프이며, $\lim_{x \to 0} 1/x^2 = \infty$이다.

  \begin{center}
  \begin{tikzpicture}
    \begin{axis}[
        axis lines = middle,
        xlabel = $x$,
        ylabel = $y$,
        xmin = -5, xmax = 5,
        ymin = -10, ymax = 10,
        grid = both,
        grid style = {line width=.1pt, draw=gray!10},
        major grid style = {line width=.2pt,draw=gray!50},
        minor tick num = 4,
        title = {$1/x^2$}
    ]

    \addplot[domain=-5:5, samples=150, thick, blue] {1/x^2};

    % \addplot[mark=*, color=red] coordinates {(0,2) (1,6)};

    \end{axis}
    \label{fig:limit-example3}
    \caption{예제 그래프}
  \end{tikzpicture}
  \end{center}
\end{example}
theorem \ref{theorem:relation-limit-one-side-limit}에 의해 $\lim_{x \to a^{+}} f(x) = \pm\infty$와 $\lim_{x \to a^{+}} f(x) = \pm\infty$도 존재함을 알 수 있다.

\section{수직 점근선}
첫 시간, 교수님께서 수평선을 그리며 말씀하셨다: `수직선 위에'. 수평선을 그리며 왜 수직선이라고 말씀하는지 물어본 학생이 있다고 한다.(물론 나는 아니다, 난 그렇게 궁금한 것이 많은 학생은 아니다 :) 교수님께서도 친절히 설명해 주셨고, 위키피디아\footnote{https://ko.wikipedia.org/wiki/\%EC\%88\%98\%EC\%A7\%81\%EC\%84\%A0}에도 친절히 적혀 있는데, 수학에서 `수직선'은 2가지 뜻을 가진 동음이의어라고 한다. 1) 두 직선 등에 직각으로 만나는 선. 2) 수를 직선에 대응시켜 표시한 직선.(친구가 `학생회-관'으로 끊어 읽던데, 마치 이런 느낌이려나? `수직-선'과 `수-직선') 다음에 정리한 `수직 점근선'의 수직은 vertical을 뜻한다. `직각의 관계'를 나타내는 수직이라고 할 수 있겠다.
\begin{definition}[수직 점근선]
  다음중 1개라도 만족하면 직선 $x = a$를 곡선 $y=f(x)$에 대한 수직 점근선\textsuperscript{vertical asymptote}이라 한다:
  \begin{multicol}[2]
  \begin{align}
    \lim_{x \to a} f(x) = \infty \\
    \lim_{x \to a} f(x) = -\infty \\
    \lim_{x \to a^{-}} f(x) = \infty \\
    \lim_{x \to a^{-}} f(x) = -\infty \\
    \lim_{x \to a^{+}} f(x) = \infty \\
    \lim_{x \to a^{+}} f(x) = -\infty
  \end{align}
  \end{multicol}
\end{definition}

\begin{note}
  2가지 의문이 있다:
  \begin{enumerate}
    \item
      첫 2개의 식이 왜 있는가? 위 수직 점근선에 대한 정의는 교재를 참고했는데, 해당 교재에는 다음 6개의 식 중 1개 만이라도 만족한다면 수직 접근선이라 말한다. 첫 식이 만족하기 위해서는 3번째와 5번째 조건이 모두 만족해야 하는 것 아닌가? 마찬가지로 2번째 식이 만족하기 위해서는 4번째와 6번째 식 모두가 만족해야만 한다. 때문에 1-4번째 식만 존재하여도 정의가 성립한다. 그저 극한 표현을 많이 써보고 싶기에, 의미가 달라지지 아니하기 때문에 첫 두개의 식을 추가했으리라 유추하고 넘어간다.
    \item
      왜 곡선 $y=f(x)$인가? 직선에서는 수직 점근선이 생기지 않기 때문인듯하다.
  \end{enumerate}
\end{note}

\begin{example}
  다음 그래프 $1/(x-3)$의 수직 점근선은 $x=3$이다:
  \begin{center}
  \begin{tikzpicture}
    \begin{axis}[
        axis lines = middle,
        xlabel = $x$,
        ylabel = $y$,
        xmin = -5, xmax = 5,
        ymin = -10, ymax = 10,
        grid = both,
        grid style = {line width=.1pt, draw=gray!10},
        major grid style = {line width=.2pt,draw=gray!50},
        minor tick num = 4,
        title = {$1/(x-3)$}
    ]

    \addplot[domain=-5:5, samples=200, thick, blue] {1/(x-3)};

    % \addplot[mark=*, color=red] coordinates {(3,-5) (3,5)};
    \addplot[color=red] coordinates {(3,-10) (3,10)};

    \end{axis};
    \label{fig:limit-example4};
    \caption{예제 그래프};
  \end{tikzpicture}
  \end{center}
  $1/(x-3)$은 $1/x$를 x축 방향으로 3만큼 평행이동한 그래프이다. 때문에 수직 점근선도 $1/x$에서의 수직 점근선 $x=0$에서 3만큼 평행이동한 $x=3$이 되는 것이다.
\end{example}

\section{극한 법칙}
사실 여기도 고등학생 시절 주구장창 쓰던 법칙 아닌가? 수업 시간에도 소개하고 증명 없이 넘어간다. 마음 같아서는 모든 식을 증명해보고 프지만, 아쉽게도 1시간 듣고 3시간 정리할 열정 없다.

\begin{enumerate}
  \item
    \ref{example:limit-example}를 보면 분명 엄격한 수학적 증명과 표현은 중요해 보인다.\\$\rightarrow$ 엄격하고 수학적인 방법에 대하여 공부해보고 싶다.
  \item
    나는 수학과가 아니고 컴퓨터 공학부 학생 아닌가? 본 수업도 엄격한 수학적 증명\footnote{큰 관련이 있는지는 모르겠지만 문득 생각나서 적어본다: 어떤 책에서 보았던 내용인데, 망델브로 집합으로 유명한 망델브로는 수학적 엄격함을 중요시 하지 아니했다고 한다. 당시 프랑스의 부르바키로 부터 수학적 엄격함이 굉장히 중요시 되었던 시대였기에 프랑스 내 최고 명문 대를 자퇴하는 일도 있었다고 한다. 또한 푸앙카레는 `맞는건데 왜 증명해야해?'라고 말했다고 한다. 그냥 다양한 사람이 있구나 생각하자.}을 목적으로 하지 아니한다. 누구는 말한다, 공학의 꽃은 활용이라고.\\$\rightarrow$ 굳이 엄격한 명확성을 따져야 할까? 컴퓨터도 공부해야하고 공부해보고 싶은 점이 많은데?
\end{enumerate}

다시 돌아와서, 극한 법칙을 나열한다:
\begin{theorem}
  극한 $\lim_{x \to a} f(x)$와 $\lim_{x \to a} g(x)$가 존재한다고 가정하자. 다음 법칙이 성립한다($c$는 상수):
  \begin{align}
    &\lim_{x \to a} (f(x) + g(x)) = \lim_{x \to a} f(x) + \lim_{x \to a} g(x) \\
    &\lim_{x \to a} (f(x) - g(x)) = \lim_{x \to a} f(x) - \lim_{x \to a} g(x) \\
    &\lim_{x \to a} (cf(x)) = c\lim_{x \to a} f(x) \\
    \text{곱의 법칙: }&\lim_{x \to a} (f(x)g(x)) = \lim_{x \to a} f(x) \cdot \lim_{x \to a} g(x) \\
    \text{몫의 법칙: }&\lim_{x \to a} \frac{f(x)}{g(x)} = \frac{\lim_{x \to a} f(x)}{\lim_{x \to a} g(x)}\quad (\lim_{x \to a} g(x) \neq 0)
  \end{align}
\end{theorem}
극한 $\lim_{x \to a} f(x)$와 $\lim_{x \to a} g(x)$가 존재해야 함을 다시 한번 강조하자. 5번째 식에서 분모가 0이 되면 안된다는 점도 다시 한번 상기하면 좋겠다.

추가로 대학 수업에서 교수님이 말하지는 아니하였지만, 고등학생 때 선생님께서 강조한 내용이다: 곱의 법칙은 역이 성립하지 아니한다. i.e.,
\begin{equation}
  \lim_{x \to a} f(x) \cdot \lim_{x \to a} g(x) \notimplies \lim_{x \to a} (f(x) \dot g(x))
\end{equation}

교재의 예제 문제를 보면 풀이 방법에서 모든 과정을 무슨 법칙을 사용한 것인지 묻는 문제가 있는데, 이 과정이 중요한 것 같다. 혹여나 나중에 덧셈이 정의되지 않은 공간을 다룰지도 모르는 일이니까.

추가적인 극한 법칙:
\begin{theorem}
  $n$은 양의 정수이다:
  \begin{align}
    &\lim_{x \to a} \left(f(x)\right)^n = \left(\lim_{x \to a} f(x)\right)^n \\
    &\lim_{x \to c} c = c \\
    &\lim_{x \to a} x = a \\
    &\lim_{x \to a} x^n = a^n \\
    &\lim_{x \to a} \sqrt[n]{x} = \sqrt[n]{a}\quad (n\text{이 짝수이면 } a>0) \\
    &\lim_{x \to a} \sqrt[n]{f(x)} = \sqrt[n]{\lim_{x \to a} f(x)} \quad (n\text{이 짝수이면 } \lim_{x \to a} f(x)>0)
  \end{align}
\end{theorem}
극한 법칙을 수학적 귀납해서 유용한 식들을 만들 수 있다.

\section{직접 대입 성질}
\begin{theorem}[직접 대입 성질]
  \index{직접 대입 성질}
  $f$가 다항함수이거나 유리함수이고 $a$가 $f$의 정의역에 있으면 다음을 만족한다:
  \begin{equation}
    \lim_{x \to a} f(x) = f(a)
  \end{equation}
\end{theorem}
사실, $f$가 다항함수라고 하였다면 $a$가 어떤 실수이든 관련 없다.(고등학생 때와 마찬가지로 calculus 과목에서도 수는 허수가 아닌 실수만을 고려하는 듯 하다.) 비단, 유리함수에 경우 분모가 0일 경우 함수값이 정의되지 아니한다. e.g., 유리함수 $f(x) = 1/x$는 $x=0$에서 함숫값이 정의되지 아니하기에 0은 domain에 속하지 아니한다. 이외 domain에 속하는 부분은 근방이 연속이다.

\begin{note}
  수업 시간에도 다룬 한가지 예시를 보자:
  \begin{align}
    \lim_{x \to 1} \frac{x^2 - 1}{x - 1}& = \lim_{x \to 1} \frac{(x-1)(x+1)}{x-1}\label{eq:이름뭘로함?} \\
                                        & = \lim_{x \to 1} x + 1 \\
                                        & = \lim_{x \to 1} x + \lim_{x \to 1} 1 \\
                                        & = 2
  \end{align}
  \ref{eq:이름뭘로함?}에서 분모와 분자의 $x-1$을 약분하는데, 이는 $\lim_{x \to 1}$이기 때문에, 즉 $x$의 값이 1이 아님이 자명하기에 약분 가능한 것이다. 고등학생 시절 해설도 없는 수능에서 풀이법이 뭐가 중요하다고 이런것을 따졌겠냐만, 이젠 주의하자.
\end{note}

\section{좌-우극한을 나누어 생각하는 경우}
간단히 한줄로 정리하겠다: 극한을 구하고자 하는 지점이 함수가 나뉘는 부분이라면($\abs{\cdot}$ 또한 $x$와 $-x$로 함수가 나뉘는 부분이다.) 좌극한과 우극한을 나누어 푸는 요령을 익혀야 한다.

\section{샌드위치 정리}
\begin{theorem}
  $x$가 $a$ 근방에서($a$는 제외할 수 있음) $f(x) \leq g(x)$이고 $x$가 $a$에 접근할 때 $f$와 $g$의 극한이 모두 존재하면 다음이 성립한다:
  \begin{equation}
    \lim_{x \to a} f(x) \leq \lim_{x \to a} g(x)
  \end{equation}
\end{theorem}
`당연한 것이 아닌가?'라는 생각이 드는데 증명하기는 어렵다고 한다. 또 다르게 생각해보면, $\lim_{x \to a}$가 $a$의 좌극한 일 수도 $a$의 우극한일 수도 있는데 저것이 성립하나 생각 들지만, 극한이 존재한다는 조건 때문에 명확함을 느길 수 있다. 이를 확장한 샌드위치 정리를 (또!) 보자:

\begin{theorem}[샌드위치 정리]
  \index{샌드위치 정리}
  $f(x) \leq g(x) \leq h(x)$가 $a$의 근방에서 성립하고 $\lim_{x \to a} f(x) = \lim_{x \to a} h(x) = L$이면 다음이 성립한다:
  \begin{equation}
    \lim_{x \to a} g(x) = L
  \end{equation}
\end{theorem}
예제로 $\lim_{x \to 0}x^{2}\sin(1/x)=0$임을 보이는 문제가 나왔다. 고등학생 때도 똑같은 문제를 교과서에서 본 기억이 있다. 나도 좋은 문제라고 생각한다.

추가로 $\lim$와 식의 연결성보다 곱이 우선 순위가 높음을 알게 되었다. --- $\lim_{x \to 0}(x^{2}\sin(1/x^2))$으로 적지 않은 이 식이 올바르기를! ---

\section{연속성}
\begin{definition}[연속]
  다음을 만족하면 함수 $f$는 $x=a$에서 연속\textsuperscript{continuous}이다:
  \begin{equation}
    \lim_{x \to a} f(x) = f(a)
  \end{equation}
\end{definition}
사실 위 식은 몇가지 숨겨진 사실이 있다. 1) $a$는 $f$의 domain에 속한다. 2) $f$의 좌극한과 우극한은 서로 같다.

\begin{definition}[불연속]
  함수 $f$가 $x=a$에서 연속이 아니면 함수 $f$는 $x=a$에서 불연속\testsuperscript{discontinuous}이다.
\end{definition}
같은 날 확률 및 통계 수업에서는 countable이라는 개념이 등장하였는데, 여기서 역은 uncountable로 접두사 un을 사용한다. 이와 discontinuous가 접두사의 차이의 뉘양쓰를 보이는 좋은 예인듯 하다.

\begin{remark}
  추가로 몇가지 연속에 대한 이름을 배웠다:
  \begin{itemize}
    \item
      제거 가능한 불연속\textsuperscript{removable discontinuouity}: $a$를 제외한 $a$의 근방이 연속이지만, $a$에서 연속이 아니거나 정의되지 아니한 상황
    \item
      무한 불연속\textsuperscript{infinite discontinuous}: $a$에서 한쪽이라도 극한이 무한대로 발산하는 경우
    \item
      도약 불연속\textsuperscript{jump discontinuities}: step function인 경우
    \item
      진동 불연속\textsuperscript{oscillatory discontinuity}: 주기 함수 등에서 무한히 진동하는 경우, e.g., $\sin(1/x)$
  \end{itemize}
  \begin{table}
    \centering
    \begin{tabular}{|c|c|}
      \hline
      1종 불연속 & 2종 불연속 \\
      \hline
      제거 가능한 불연속 & 무한 불연속 \\
      도약 불연속 & 진동 불연속 \\
      \hline
    \end{tabular}
  \end{table}
\end{remark}

\begin{definition}[왼-오른쪽에서 연속]
  다음을 만족할 때 $f$는 $x=a$에서 왼쪽에서 연속\textsuperscript{continuous from the left}이다:
  \begin{equation}
    \lim_{x \to a^-} f(x) = f(a)
  \end{equation}
  반대로, 다음을 만족할 때 $f$는 $x=a$에서 오른쪽에서 연속\textsuperscript{continuous from the right}이다:
  \begin{equation}
    \lim_{x \to a^+} f(x) = f(a)
  \end{equation}
\end{definition}
사실 왼쪽과 오른쪽에서 연속이라는 말을 잘 사용하는지는 의문이다. `좌극한과 함숫값이 같다'라고 말하는 것도 길지 아니하기 때문이다.
\begin{theorem}
  다음은 동형이다:
  \begin{equation}
    \lim_{x \to a} f(x) = f(a) \Longleftrightarrow \lim_{x \to a^-} f(x) = f(a) = \lim_{x \to a^+} f(x)
  \end{equation}
\end{theorem}

\end{document}
